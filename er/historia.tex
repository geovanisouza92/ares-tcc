% Seção: História das linguagens de programação

\section{Hist\'oria das linguagens de programa\ca o}
\label{revisao:historia}

Segundo \cite{Aho08}, a partir de 1940 os primeiros computadores eletr\^onicos eram programados com sequ\^encias de 0s e 1s, que diziam para a m\'aquina exatamente o que fazer e em que ordem. Esse processo era lento, cansativo e pass\'ivel de erros.

A partir de 1950, surgiram linguagens mais intelig\'iveis para o homem, por\'em no in\'icio, eram apenas mnem\^onicos para as instru\co es de 0s e 1s. Posteriormente surgiram \emph{macros} que resumiam sequ\^encias de mnem\^onicos.

Na segunda metade da d\'ecada de 1950, surgiram as linguagens Fortran, para computa\ca o cient\'ifica, Cobol, para processamento de dados comerciais e Lisp, para a computa\ca o simb\'olica. O objetivo de tais linguagens, chamadas de alto n\'ivel, era de facilitar a constru\ca o de programas cient\'ficos, comerciais e simb\'olicos. Seu sucesso foi t\~ao grande que continuam a ser usadas at\eh\ hoje.

Nas d\'ecadas seguintes, surgiram diversas linguagens com recursos inovadores que tornaram a programa\ca o mais f\'acil e produtiva, conclui \cite{Aho08}.
