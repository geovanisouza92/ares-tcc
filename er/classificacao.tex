% Seção: Classificação das linguagens de programação

\section{Classifica\ca o das linguagens de programa\ca o}
\label{revisao:classificacao}

\cite{Aho08}, diz que as linguagens de programa\ca o podem ser classificadas quanto a gera\ca o e paradigma.

\cite{MacLennan}, afirma que as linguagens se dividem em 5 gera\co es, agrupando os paradigmas da orienta\ca o a objetos, funcional e l\'ogico como sendo a 5ª gera\ca o.

\subsection{Por gera\ca o}

\cite{Aho08} e \cite{wiki:lingprog} classificam as linguagens de programa\ca o de forma semelhante:

\begin{description}
\item[1ª gera\ca o] Linguagem de m\'aquina ou bin\'aria, que \eh\ a sequ\^encia de 0s e 1s.
\item[2ª gera\ca o] Linguagem de montagem, simb\'olica ou \emph{assembly}.
\item[3ª gera\ca o] Linguagem de alto-n\'ivel, procedurais, como Fortran, Cobol, Lisp, C, C++, C\# e Java, muitas das quais s\ao utilizadas at\eh\ hoje.
\item[4ª gera\ca o] Linguagem aplicativa ou espec\'ifica de dom\'inio, como NOMAD\footnote{Banco de dados relacional e linguagem para gera\ca o de relat\'orios, muito utilizada nas d\'ecadas de 70 e 80.}, SQL e Postscript.
\item[5ª gera\ca o] Linguagens l\'ogicas e funcionais, como ML, Haskell, Lisp e Prolog.
\end{description}

A linguagem de 6ª gera\ca o, segundo \cite{wiki:lingprog}, \eh\ aquela utilizada para descrever e construir redes neurais\footnote{Uma refer\^encia completar sobre o assunto pode ser encontrada em \cite{wiki:redesneurais}.} e aut\^omatos.

\subsection{Por paradigma}

\cite{Aho08} e \cite{wiki:lingprog} classificam dois paradigmas principais:

\begin{description}
\item[Imperativo] Onde o programa diz exatamente \emph{como} fazer cada opera\ca o. Agrupa as linguagens ditas procedurais (Fortran e BASIC, por exemplo), estruturadas (C, Pascal e Algol, por exemplo), orientadas a objeto (PHP, C++, C\#, Java, Ruby e Python, por exemplo) e linguagens para computa\ca o distribu\'ida, como Ada.
\item[Declarativo] Quando o programa diz \emph{o que} fazer, que podem ser funcionais (Lisp, Scheme e Haskell, por exemplo) ou l\oh gicas (como Prolog e G\"odel).\\[2.5cm]
\end{description}
