% Descrição do sistema existente
\chapter{Descri\ca o do Sistema Existente}

\label{pro:descricao-existente}

Como sistema existente, tomaremos a plataforma Java, pois representa uma ferramenta bem estabelecida no contexto.

A linguagem de programa\ca o Java foi criada por James Gosling na Sun Microsystems, em 1995.

Segue o paradigma da orienta\ca o a objetos, \eh\ estruturada e imperativa\footnote{Linguagem imperativa \eh\ aquela em que o programador diz exatamente \emph{como} deve ser feito. Linguagem declarativa \eh\ aquela em que o programador diz \emph{o que} ele quer, n\ao\ se importando em como o programa dever\ah\ realiz\ah-lo, como a linguagem SQL, por exemplo.}.

Foi influenciada pelas linguagens Ada, C++, C\#\footnote{Apenas nas vers\~oes mais recentes}, Eiffel, Modula-3, Object Pascal, Smalltalk, entre outras.

Influenciou as linguagens C\#\footnote{H\ah\ de se notar diversas semelhan\c{c}as entre as plataformas Java e .NET, al\eh m da linguagem.}, Clojure, D, ECMAScript\footnote{ECMAScript \eh\ o padr\ao\ de linguagens de script, definida pela organiza\ca o ECMA, dentre as quais podemos citar JavaScript, ActionScript e C\#.}, Groovy, JavaScript, PHP, Scala e outras.

Sua tipagem\footnote{Tipagem representa o modelo de tipos de dados fornecido pela linguagem e como a linguagem o respeita} \eh\ est\ah tica e forte, definida ainda na etapa de compila\ca o.

Seus compiladores mais conhecidos s\ao\ o \emph{javac}, padr\ao\ da plataforma e o \emph{GCJ}\footnote{GNU Compiler for Java, parte do GCC.}.

Seu licenciamento de uso \eh\ o GPL (GNU General Public License) e suas diretrizes e novas funcionalidades definidas pela organiza\ca o \emph{Java Community Process}\footnote{Os processos de implementa\ca o da linguagem podem ser acompanhados em \emph{http://www.jcp.org/}}.
